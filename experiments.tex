\section{Experimental results}\label{sec:experiments}

Having established our theoretical framework, we now provide experimental
results on the egalitarianism of various cryprocurrencies. Our experiments
utilize the \emph{egalitarian curve} definition of section~\ref{sec:definition}
in order to concretely confirm --- or disprove --- the egalitarianism claims of
some of the major, both proof-of-work and proof-of-stake, cryptocurrencies.

\noindent\textbf{Proof-of-work}

\noindent
We have experimentally analyzed the egalitarianism of the following
proof-of-work coins: Bitcoin, Litecoin, Ethereum, and Monero. These
cryptocurrencies act as a representative sample among the thousands of existing
cryptocurrencies. Bitcoin is the largest and most successful cryptocurrency by
market cap. Litecoin is the first cryptocurrency aimed at becoming more
egalitarian by replacing Bitcoin's SHA256 work function with scrypt, a more
memory-hard function. Ethereum is one of the most promising alternative
cryptocurrencies, the first to support smart contracts, and the second largest
by market cap; its work function is different from both Bitcoin and Litecoin.
Finally, Monero is special with claims of strong egalitarianism due to its
memory-hard mining function, Cryptonight. Furthermore, its protocol is often
updated to maintain egalitarianism\todo{reference for this?}.

As expected, our experiments show that Bitcoin is the least egalitarian of the
four, with Ethereum following next. Litecoin is more egalitarian than both, due
to its use of scrypt. Finally, Monero is the most egalitarian among the
proof-of-work coins we have studied.

\dionyziz{Revise above results based on experimental data.}

For our experimental setting, we worked as follows. First, we collected
empirical data which describe the available mining hardware options in the
market. For each machine choice, we determined the cost of investment. This is
comprised of its initial price (in USD) as well as its energy cost of operation
(in Watts). The cost of operation was translated to USD per hour by considering
the electricity cost of KWh. As a reference, we used the lowest average KWh
cost in the United States, \ie $0.08\$$ per KWh\footnote{See:
\url{https://en.wikipedia.org/wiki/Electricity_pricing} [Online: December
2018]}. We remark that this reference electricity cost is an estimation which
can vary depending on the country of operation, contractual miner agreements
with the power grid etc. Althouth we do extract exact numbers using this
assumption, we stress that the qualitative difference of egalitarianism across
cryptocurrencies, \ie the shape of their egalitarian curve and the order of
egalitarianism, remains the same regardless of the electricity cost.

Second, we use the reported hash rate of each mining hardware machine to
extract an expectation of the freshly mined coins it would generate per hour,
if it were to run continuously. This expectation is taken over the randomness
of all honest blockchain protocol executions. As such, each party is awarded
block rewards in proportion to their computational power. The difference
between revenue per unit of time and cost of operation per unit of time
produces an \emph{income rate}, which is measured in USD per hour.

For our experiments, we use a duration of investment of $d = 1$ year. Although
this choice is arbitrary, it corresponds to the usual definition of ROI in
traditional finance. Again this parameter does not make a qualitative
difference in our metrics.

To find the optimal investment strategy, we employ the following algorithm. The
initial available capital is allocated to an --- upfront --- technology
investment, in which an integer instance of the ``Knapsack'' problem
\cite{mathews1896partition} is solved to optimize the total cashflow.
Subsequently, as long as the cashflow is positive, the purchased machines
operate for the indicated total duration, reinvesting part of the freshly
minted coins in electricity costs, in order to generate more coins. Eventually,
this strategy produces an income of freshly generated coins, which have not
been spent and are reported as the strategy's income.

To calculate our concrete numbers, we employ the constants shown in
Table~\ref{tbl:work-constants}. We use the expected block generation rates for
each cryptocurrency, as well as the reward per block, token price, and mining
difficulty at the time of writing, all of which we assume remain constant.
While we cannot predict future prices and difficulty, it is worth noting that
these do not affect the relative comparison of cryptocurrencies in terms of
egalitarianism \todo{not necessarily true, see next sentence}.  Although small
changes in the constants' values will not affect the relative comparison, we
note that extreme changes may reorder the currencies in terms of
egalitarianism. For example, consider the case of Bitcoin where all mining
products, except a relatively cheap one, are unprofitable, a plausible scenario
in case \eg the price of Bitcoin drops significantly. In this case, Bitcoin
could become as egalitarian, or even more, as \eg Litecoin. Thus, identifying
the ``breaking points'' of egalitarianism for each cryptocurrency, compared to
others, is an interesting problem which needs be considered in future work.

\begin{table}
  \centering
  \resizebox{\textwidth}{!}{%
    \begin{tabular}{|c|c|c|c|c|c|c|c|}
      \hline
      Variable & Description & Unit & BTC & ETH & LTC & XMR & DCR\\
      \hhline{|=|=|=|=|=|=|=|=|}
      $d$ & duration of investment & years & \multicolumn{5}{c|}{$1$} \\
      \hline
      $\ec$ & electricity cost & USD / kWh & \multicolumn{5}{c|}{$0.08$} \\
      \hline
      $\bgr(c)$ & block generation rate & blocks / s & 1 / 600 & 1 / 14.7 & 1 / 150 & 1 / 122 & 1 / 298\\
      \hline
      $\thr(c)$ & total hash rate & Th / s & 34,727,437 & 179.50374 & 174.537 & 0.00033859 & 178,760\\
      \hline
      $\br(c)$ & reward per block & tokens & 12.5 & 3 & 25 & 3.37 & 11.38\\
      \hline
      $\tp(c)$ & token price & token / USD & 4,074.25 & 126.12 & 32.10 & 47.27 & 18.62\\
      \hline
    \end{tabular}
  }
  \caption{The parameters to our proof-of-work mining simulations. Some depend on the cryptocurrency $c$.}
  \label{tbl:work-constants}
\end{table}

\dionyziz{confirm that changing the total hash rate maintains curve shapes and argue about it in the main text}

Using the constant values from table\ref{tbl:work-constants} we now define the
computation of egalitarianism in our experiments.  Let $\mathbb{M}$ denote the set of
all available mining machines. For each machine $m \in \mathbb{M}$, our empirically
collected data specifies the following parameters:
\begin{inparaenum}[i)]
    \item the energy consumption rate $\ecr(m)$ in Watts,
    \item an initial cost of purchase $\ic(m)$ in USD, and
    \item a hash rate $\hr(m)$ in Th per second.
\end{inparaenum}
Given the above, we can now calculate the expected income rate per hour
$\mathbb{E}[\ir(m)]$ for a given machine $m$ and a cryptocurrency $c$. In the
following equation, the first part identifies the income per hour, \ie the
amount of tokens (denominated in USD) which the machine produces per second
multiplied by $3600$ ($60$ seconds per minute times $60$ minutes per hour),
whereas the second part of the equation identifies the electricity cost, \ie
the product of the consumed electricity multiplied by the price of a single
KWh:

\[
\mathbb{E}[\ir(m)] = 3600 \cdot \frac{\hr(m)}{\thr(c)} \cdot \br(c) \cdot \bgr(c) \cdot \tp(c) - \ecr(m) \cdot \ec
\]

There are many possible configurations for technology investments. Each
configuration comprises of a number of copies $n \in \mathbb{N}$ of every
machine type $m \in \mathbb{M}$. Therefore, we define each configuration as
$\overline{m} \subseteq \mathbb{M} \times \mathbb{N}$, with
total initial cost of investment for such configuration being:

\[
  \ic(\overline{m}) = \sum_{(m, n) \in \overline{m}}{n \cdot \ic(m)}
\]

The above figure is given in USD per hour and, since the initial capital should
suffice to buy the machines of the configuration, we require that the following
holds:

\[
    \ic(\overline{m}) \leq \mathcal{C}
\]

where $\mathsf{C}$ is the initial available capital at the beginning of the simulation.

Now, in order to identify the \emph{optimal} strategy's reported net income for the
duration $d$, we iterate over all possible machine configurations, for which
the above inequality holds, and choose the one with the maximum returns:

\[
  B_\textsc{OPT}(v)
  =
  \max{
    \{
      \sum_{(m, n) \in \overline{m}}
      {d\mathbb{E}[\ir(m)]}:
      \overline{m} \subseteq \mathbb{M} \times \mathbb{N}
      \land
      \ic(\overline{m}) \leq v
    \}
  }
\]

As the simulation parameters are many and diverse, in order to allow others to
run the experiments with different values, as well as for reasons of
reproducibility and falsifiability, we openly release our mining investment
optimizer as well as our data for public use\ifanonymous\footnote{
  The link to our mining investment calculator and our mining hardware data,
  which are available under an open source license, has been redacted from this
  version for anonymity purposes.
}\else\footnote{
  Our mining investment calculator and our mining hardware data are available
  under the MIT license and a Creative Commons 4.0 Attribution License
  respectively at \url{https://github.com/decrypto-org/egalitarianism}.
}\fi.

The egalitarianism of Bitcoin, Ethereum, Litecoin and Monero are shown in
Figures~\ref{fig:btc_dp_10K_12_months}, ~\ref{fig:eth_dp_10K_12_months},
~\ref{fig:ltc_dp_10K_12_months}, and~\ref{fig:xmr_dp_10K_12_months}
respectively.

Finally, Decred is a hybrid proof-of-work/proof-of-stake cryptocurrency, in
which block generation is a collaboration between miners and minters.
Specifically, each block which is mined via proof-of-work needs to be
``vouched'' for by a certain number of minters, who give it a vote of
confidence. Both the miners and the minters whoc participate in block
generation are rewarded. An investor can therefore choose to participate in
Decred by either investing in mining hardware and performing proof-of-work, or
by purchasing stake and performing proof-of-stake (or a combination thereof).
We note that the choice of whether to mine or mint Decred is not always clear.
While mining may be more profitable for a certain initial capital, it can also
carry various risks. For instance, if the difficulty increases, the mining
hardware may be rendered inefficient and also hard to sell. Proof-of-work also
carries the operational overhead discussed in Remark~\ref{rmk:pow-scale}. On
the other hand, stake can always be sold, although the price may fluctuate, and
carries negligible operational overhead. As the decision between the two is not
obvious, we analyze both strategies independently. The egalitarianism of
proof-of-work mining for Decred is shown in
Figure~\ref{fig:xmr_dp_10K_12_months}.

It is evident from all figures that the ROI is ``capped'' by a maximum value,
which is observed in specified intervals. Indeed, this value identifies the
ROI of the \emph{best available} machine and is in line with Theorem~\ref{thm:sybil}. In other words, as long as an
investor is able to buy the machine which returns the most profits, then they
observe the best possible ROI. In case an investor does not have enough capital
to buy the best mining product, they may buy a less profitable machine and
observe less, though still positive, ROI. This observation explains the small
spikes in ROI which may be seen \eg in Bitcoin's figure for capital in the
range $[0, 2000]$. Also, in case the capital is \emph{more} than the cost of
the machine, then the remaining capital is effectively discarded. Therefore,
although two investors $A, B$ may start with initial capital $\mathcal{C}_A <
\mathcal{C}_B$, if their returns, in absolute terms, are the same, then the ROI
of $B$ will be smaller as a percentage compared to the ROI of $A$. This
observation explains the decrease in ROI after the spikes. Finally, we observe
that, as the capital increases, the ROI converges to a set value. This is
explained by the fact that the ``discarded'' capital, \ie the capital which
cannot be invested in mining hardware, is a significantly smaller percentage of
the total capital for large investments.

\begin{figure}
  \placesubfigure{btc_dp_10K_12_months.pdf}{fig:btc_dp_10K_12_months}{The egalitarianism of Bitcoin.}{0.75}{0.5}
  \placesubfigure{eth_dp_10K_12_months.pdf}{fig:eth_dp_10K_12_months}{The egalitarianism of Ethereum.}{0.75}{0.5}
  \placesubfigure{ltc_dp_10K_12_months.pdf}{fig:ltc_dp_10K_12_months}{The egalitarianism of Litecoin.}{0.75}{0.5}
  \placesubfigure{xmr_dp_10K_12_months.pdf}{fig:xmr_dp_10K_12_months}{The egalitarianism of Monero.}{0.75}{0.5}
  \placesubfigure{dcr_dp_10K_12_months.pdf}{fig:dcr_dp_10K_12_months}{The egalitarianism of \emph{proof-of-work} Decred.}{0.75}{0.5}
  \caption{Proof-of-work egalitarian curves}
  \label{fig:egalitarian_curves_pow}
\end{figure}

\noindent\textbf{Proof-of-stake}

\noindent
We now analyze the proof-of-stake egalitarianism in two settings. First, we
consider pure proof-of-stake, which can be applied on top of a protocol like
Ouroboros. In this case, \emph{pure} is in opposition to \emph{delegated}
proof-of-stake, a setting where the stakeholders are required to delegate their
stake to other players, namely ``stake pools'' and is deployed in
cryptocurrencies such as EOS, Bitshares, and others. Second, we consider the
case of minting Decred via its proof-of-stake mechanism.

The egalitarian curve for \emph{staking} Decred is illustrated in
Figure~\ref{fig:decred-stake}.
% The curve is almost perfectly egalitarian.
As mentioned above, Decred is an opt-in staking cryptocurrency, where staking
occurs by purchasing so-called \emph{tickets}. Since the price of a ticket is
quantized, egalitarianism is harmed for capitals which are not multiples of
ticket prices. However, one can see that the envelope of maxima of this curve
is perfectly egalitarian. Perfect egalitarianism is also achieved when one
considers the distribution $\mathcal{D}$ of initial capitals that are multiples
of the ticket price.

In the case of pure proof-of-stake, every coin has the same probability of
being chosen for eligibility. When a coin is eligible for block generation, its
owner can create a block by providing a proof of ownership of the chosen coin.
Consider the case of a cryptocurrency with $N$ coins in circulation. When a
block needs to be created, a coin is chosen at random from the set of $N$
coins. Therefore, each coin may be chosen with $1 \over N$ probability. Then
the address which owns the chosen coin, in other words the stakeholder which
controls this coin, is eligible to generate a block and receive the block
rewards associated. In our experiments, we assume that every block is
associated with a constant reward, which pertains to newly minted coins.
Furthermore, since computational power does not affect the rate of block
production, it is reasonable to assume that both the electricity and the
hardware equipment's price is constant for all users, regardless of stake
accumulation, so all users can participate using --- relatively --- cheap
resources.

Figure~\ref{fig:pure-pos-stake} depicts the simulation of a pure proof-of-stake
system. In this case, the users pay a set fee which represents the hardware and
electricity cost, whereas the rest of the initial capital is used to buy stake.
It is evident from the figure that this system is the closest to perfect
egalitarianism compared to the proof-of-work and hybrid cases.

\begin{figure}
  \placesubfigure{decred-stake.pdf}{fig:decred-stake}{The egalitarianism of Decred's proof-of-stake mechanism.}{0.75}{0.5}
  \placesubfigure{pure-pos.pdf}{fig:pure-pos-stake}{The egalitarianism of \emph{staking} in a pure proof-of-stake cryptocurrency.}{0.75}{0.5}
  \caption{Proof-of-stake egalitarian curves}
  \label{fig:egalitarian_curves_pos}
\end{figure}

\noindent\textbf{Summary}

\noindent
Our findings are summarized in Table~\ref{tbl:egalitarianism}. We find that
Bitcoin is the least egalitarian, followed by Ethereum, Litecoin, and Monero.
The latter two are the most egalitarian due to their use of scrypt and
Cryptonight respectively. Decred is a hybrid coin which can both be staked and
mined. Mining with Decred has egalitarianism which stands between Ethereum and
Litecoin. However, the most egalitarian coins involve staking. Decred staking,
due to its quantized ticket pricing, is only approximately egalitarian. Pure
proof-of-stake, which allows continuous staking, is \emph{almost perfectly
egalitarian}, its small divergence from perfect egalitarianism resulting from
the --- small --- set fee required to buy a regular machine and use it to join
the network and participate in the protocol.

\dionyziz{Carefully reread and revise findings of above paragraph}

\begin{table}
  \centering
  \begin{tabular}{|c|c|c|}
    \hline
    Name & Consensus mechanism & Egalitarianism \\
    \hline
    Bitcoin &  Proof-of-work &  -0.03413823178946599 \\
    \hline
    Ethereum & Proof-of-work &  -0.00610617267350942 \\
    \hline
    Litecoin & Proof-of-work &  -0.00027182201930516 \\
    \hline
    Monero &   Proof-of-work &  -0.00225939513462083 \\
    \hline
      Decred & \begin{tabular}{c} Proof-of-work \\ Proof-of-stake \end{tabular} & \begin{tabular}{c} -0.37581208342736927 \\ -0.00034828009234172 \end{tabular}  \\
    \hline
    Pure PoS & Proof-of-stake & -0.00000000178577788 \\
    \hline
  \end{tabular}
  \caption{The egalitarianism of the various cryptocurrencies we studied.}
  \label{tbl:egalitarianism}
\end{table}
