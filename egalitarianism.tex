%!TEX encoding = UTF-8 Unicode

\newif\ifanonymous
\newif\ifshort
\newif\iflncs
\lncstrue
% \anonymoustrue
% \shorttrue

\documentclass[a4paper,UKenglish,cleveref, autoref]{oasics-v2019}
\usepackage{preamble}

\usepackage{longtable}

% \bibliographystyle{plainurl}% the mandatory bibstyle


%!TEX root = egalitarianism.tex

\title{Cryptocurrency Egalitarianism:\protect\\A Quantitative Approach} %TODO Please add

\titlerunning{Cryptocurrency Egalitarianism}%optional, please use if title is longer than one line

    \author{Dimitris Karakostas}{University of Edinburgh, IOHK}{dimitris.karakostas@ed.ac.uk}{}{}%TODO mandatory, please use full name; only 1 author per \author macro; first two parameters are mandatory, other parameters can be empty. Please provide at least the name of the affiliation and the country. The full address is optional

    \author{Aggelos Kiayias}{University of Edinburgh, IOHK}{akiayias@inf.ed.ac.uk}{}{}
    \author{Christos Nasikas}{University of Athens, ``Athena'' Research Center}{xnasikas@di.uoa.gr}{}{}
    \author{Dionysis Zindros\footnote{Corresponding author}}{University of Athens, IOHK}{dionyziz@di.uoa.gr}{}{}

    \authorrunning{Karakostas et al.}%TODO mandatory. First: Use abbreviated first/middle names. Second (only in severe cases): Use first author plus 'et al.'

\Copyright{Sara Tucci-Piergiovanni et al.}%TODO mandatory, please use full first names. LIPIcs license is "CC-BY";  http://creativecommons.org/licenses/by/3.0/

\ccsdesc[100]{Security and privacy~Economics of security and privacy}
% \ccsdesc[100]{Security and privacy}%TODO mandatory: Please choose ACM 2012 classifications from https://dl.acm.org/ccs/ccs_flat.cfm

\keywords{blockchain, egalitarianism, cryptocurrency, economics, proof-of-work, proof-of-stake}%TODO mandatory; please add comma-separated list of keywords

\category{}%optional, e.g. invited paper

\relatedversion{}%optional, e.g. full version hosted on arXiv, HAL, or other respository/website
%\relatedversion{A full version of the paper is available at \url{...}.}

\supplement{}%optional, e.g. related research data, source code, ... hosted on a repository like zenodo, figshare, GitHub, ...

\nolinenumbers

\ifanonymous\else
\acknowledgements{Research partially supported by H2020 projects PRIVILEDGE \#780477 and MHMD \#732907.}
\fi

%Editor-only macros:: begin (do not touch as author)%%%%%%%%%%%%%%%%%%%%%%%%%%%%%%%%%%
\EventEditors{Vincent Danos, Maurice Herlihy, Maria Potop-Butucaru, Julien Prat, and Sara Tucci-Piergiovanni}
\EventNoEds{5}
\EventLongTitle{International Conference on Blockchain Economics, Security and Protocols (Tokenomics 2019)}
\EventShortTitle{Tokenomics 2019}
\EventAcronym{Tokenomics}
\EventYear{2019}
\EventDate{May 6--7, 2019}
\EventLocation{Paris, France}
\EventLogo{}
\SeriesVolume{71}
\ArticleNo{5}
%%%%%%%%%%%%%%%%%%%%%%%%%%%%%%%%%%%%%%%%%%%%%%%%%%%%%%

\begin{document}

\maketitle
\begin{abstract}
Since the invention of Bitcoin one decade ago, numerous cryptocurrencies have
sprung into existence. Among these, proof-of-work is the most common
mechanism for achieving consensus, whilst a number of coins have adopted
``ASIC-resistance'' as a desirable property, claiming to be more
``egalitarian,'' where egalitarianism refers to the power of each coin
to participate in the creation of new coins.  While proof-of-work
consensus dominates the space, several new cryptocurrencies
employ alternative consensus, such as proof-of-stake
in which block minting opportunities are
based on monetary ownership.  A
core criticism of proof-of-stake revolves around it being less egalitarian by
making the rich richer, as opposed to proof-of-work in which everyone can
contribute equally according to their computational power. In this paper,
we give the first quantitative definition of a cryptocurrency's
\emph{egalitarianism}. Based on our definition, we measure the egalitarianism of
popular cryptocurrencies that (may or may not) employ ASIC-resistance,
among them Bitcoin, Ethereum, Litecoin, and Monero. Our simulations show,
as expected, that ASIC-resistance increases a cryptocurrency's
egalitarianism.  We also measure the egalitarianism of a stake-based
protocol, Ouroboros, and a hybrid proof-of-stake/proof-of-work cryptocurrency, Decred. We show
that stake-based cryptocurrencies, under correctly selected parameters, can be perfectly egalitarian, perhaps contradicting folklore belief.
\end{abstract}

\import{./}{introduction.tex}
\import{./}{related.tex}
\import{./}{preliminaries.tex}
\import{./}{definition.tex}
\import{./}{experiments.tex}
\import{./}{conclusion.tex}
\iflncs
\bibliographystyle{plain}
\else
\bibliographystyle{plainurl}
\fi
\bibliography{pubs,abbrev3,crypto}

\newpage
\import{./}{appendix.tex}

\end{document}
