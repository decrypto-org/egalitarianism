%!TEX root = egalitarianism.tex

\begin{abstract}
Since the invention of Bitcoin one decade ago, numerous cryptocurrencies have
sprung into existence. Among these, ``proof-of-work'' is the most common
mechanism for achieving consensus, whilst a number of coins have adopted
``ASIC-resistance'' as a desirable property, claiming to be more
``egalitarian'', \ie providing \emph{equal} opportunities to both poor and
rich players to participate in the creation of coins. While proof-of-work
consensus algorithms dominate the space, several new cryptocurrencies
employ alternative consensus algorithms, such as ``proof-of-stake''.  A
core criticism of proof-of-stake, in which block minting opportunities are
based on monetary ownership, revolves around it being less egalitarian by
making the rich richer, as opposed to proof-of-work in which everyone can
contribute equally according to their computational power. In this paper,
we give the first quantitative definition of a cryptocurrency's
egalitarianism. Based on our definition, we measure the egalitarianism of
popular cryptocurrencies that (may or may not) employ ASIC-resistance
strategies, among them Bitcoin, Ethereum, and Monero. Our simulations show,
as expected, that ASIC-resistance increases a cryptocurrency's
egalitarianism.  We also measure the egalitarianism a popular
stake-based protocols, Ouroboros, and a proof-of-stake cryptocurrency, Decred. We show that stake-based
cryptocurrencies are perfectly egalitarian under our definition, perhaps
contradicting folklore belief.
\end{abstract}
