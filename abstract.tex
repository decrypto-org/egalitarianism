\begin{abstract}
Since the invention of Bitcoin one decade ago, numerous
cryptocurrencies have sprung into existence. Among these, many newer
proof-of-work-based coins have adopted ``ASIC-resistance'' as a
desirable property, claiming to be more ``egalitarian.'' While
proof-of-work consensus algorithms dominate the space, several new
cryptocurrencies employ other consensus algorithms such as
``proof-of-stake''. Proof-of-stake, in which block minting
opportunities are based on monetary ownership, has been criticized for
being less egalitarian because it makes the rich richer, as compared
to proof-of-work in which everyone can contribute equally based on
their computational power. In this paper, we give the first
proposal for a quantitative definition of a cryptocurrency's
egalitarianism. Based on our definition, we measure the egalitarianism
of popular cryptocurrencies that employ (or not) ASIC-resistance
strategies, among them Bitcoin, Ethereum and Monero. As expected, we
find that ASIC-resistance increases a cryptocurrency's egalitarianism.
We also measure the egalitarianism of two popular stake-based
cryptocurrencies, Decred and Cardano. We show that stake-based
cryptocurrencies are perfectly egalitarian under our definition,
perhaps contradicting folklore belief.
\end{abstract}
