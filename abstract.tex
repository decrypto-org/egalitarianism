%!TEX root = egalitarianism.tex
\begin{abstract}
Since the invention of Bitcoin one decade ago, numerous cryptocurrencies have
sprung into existence. Among these, proof-of-work is the most common
mechanism for achieving consensus, whilst a number of coins have adopted
``ASIC-resistance'' as a desirable property, claiming to be more
``egalitarian'', where egalitarianism refers to the power of each coin
to participate in the creation of new coins.  While proof-of-work
consensus dominates the space, several new cryptocurrencies
employ alternative consensus, such as proof-of-stake
in which block minting opportunities are
based on monetary ownership.  A
core criticism of proof-of-stake revolves around it being less egalitarian by
making the rich richer, as opposed to proof-of-work in which everyone can
contribute equally according to their computational power. In this paper,
we give the first quantitative definition of a cryptocurrency's
\emph{egalitarianism}. Based on our definition, we measure the egalitarianism of
popular cryptocurrencies that (may or may not) employ ASIC-resistance,
among them Bitcoin, Ethereum, Litecoin, and Monero. Our simulations show,
as expected, that ASIC-resistance increases a cryptocurrency's
egalitarianism.  We also measure the egalitarianism of a stake-based
protocol, Ouroboros, and a hybrid proof-of-stake/proof-of-work cryptocurrency,
Decred. We show that stake-based cryptocurrencies, under correctly selected
parameters, can be perfectly egalitarian, perhaps contradicting folklore belief.
\end{abstract}
