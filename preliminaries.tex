\section{Preliminaries}\label{sec:preliminaries}

The leader election is a central part of each blockchain consensus algorithm. We
now give an overview of the details of proof-of-work mining and proof-of-stake
minting, as they are central in understanding the difference in egalitarianism
between the two.

\noindent\textbf{Proof-of-work.}
In
the proof-of-work case, miners attempt to solve the proof-of-work inequality. Their
mining hardware is provided with two constants $\textsc{previd}$ and $\textsc{data}$
related to blockchain maintenance. It then
brute-force searches for some string \textsc{nonce} such that
$H(\textsc{previd} || \textsc{data} || \textsc{nonce}) \leq T$. Here, $T$ is a
(relatively) small number called
the \emph{difficulty target}, which typically remains constant for periods of several
consecutive blocks called \emph{epochs} (for example, in Bitcoin, epochs are
2016 blocks long). In this inequality, $H$ is a
cryptographically secure function. Because the search for solutions is
brute-forced, the expected number of solutions found by a given miner in a given
timeframe is proportional to the number of evaluations of the hash function $H$
she can obtain in this timeframe.

\dionyziz{Verify that modifying the difficulty retains the shape of our curves}

The number of such evaluations is one of the several critical parameters to consider when purchasing
mining hardware. The other important parameters are the cost of purchase for a mining unit,
as well as its electricity consumption, in Watts.
There exist various tiers of mining hardware which have
vastly different performance. These tiers are CPU miners, GPU miners, FPGA miners,
and specialized hardware miners in the form of ASICs. It makes a vast difference in
terms of hashing rate and, in turn, return on investment, despite the pricing of
the respective hardware not having a similarly vast difference in cost of
purchase. As a motivating indicative example, at the time of writing, the mining hardware ``Whatsminer
M10'' produced by the company ``MicroBT'' has a cost of purchase of $\$1,448.00$ per unit and produces $\$0.039916$ per hour
of operation in net gains when electricity costs are subtracted from profits. On
the other hand, the mining hardware ``8 Nano Pro'' produced by the company ``ASICMiner'' has a cost of purchase of
$\$47,500.00$ per unit, but produces $\$0.195327$ per hour of operation in net gains, close to five times the net gains per hour of its cheaper competitor.
Thus, if one can afford to purchase the more expensive hardware, each of their
subsequent dollar invested in electricity is worth more in terms of freshly mined coin returns.

It
has long been folklore knowledge in the blockchain community that, if a
memory-hard function is employed as the proof-of-work mining hash function, then
mining becomes more egalitarian and such differences are not so pronounced. This intuition is correct, because
it is difficult to construct specialized hardware for memory-hard functions, and
hence the only way to scale mining operations is by purchasing more hardware of
the cheap kind, hence scaling returns proportionally to investments. For
example, there are currently no ASICs available for Monero mining. From a theoretical computer
science point of view, memory-hard functions have been formalized in the line of
work studying pebbling games. We are the first to confirm this correspondence between the
memory-hardness of proof-of-work hash functions and the economics of mining.

\begin{remark}[Proof-of-work at scale]\label{rmk:pow-scale}
We only analyze the scaling of the economics of proof-of-work mining with
respect to hardware and electricity costs. There are a multitude of factors that
play important roles for mining operations at scale. Such examples include rental
costs for the operation of hardware, machine cooling costs, maintenance cost
for mining machines that break down which requires specialized workers to fix
them, as well as different tiers of electricity cost when energy is purchased in
bulk. As is common in economies of scale, these relative costs are reduced when
one scales their operations. These additional costs are similar for all
proof-of-work cryptocurrencies and thus do not affect relative comparisons
between them. We also remark that we analyze mining costs for small capital
investments. If larger capital (above a few millions) is available,
corporations can develop their own specialized hardware which
gives them a competitive advantage and which they treat as a trade secret.
These details will make our comparison in favour of proof-of-stake only
\emph{more pronounced}, as proof-of-stake operations do not incur any
of these costs which can be optimized at scale and do not lend themselves to
specialized mining hardware research. We leave the analysis and calculation
of egalitarianism under these parameters to future work.
\end{remark}

\noindent\textbf{Proof-of-stake.}
In the proof-of-stake case, a minter is selected in proportion to the stake they
hold, which is to say proportionally to the amount of money they own. There are
two flavours of this process. In the first flavour, all money owned automatically participates in the leader election
process. This is the case for the various variations of Ouroboros
and Ethereum's Casper~\cite{buterin2017casper}. In the second flavour, the stake has to opt-in to participate in
the election by a special process, such as purchasing a so-called \emph{ticket}. This is the case for cryptocurrencies such as
Decred. Among those participating in the election, a leader is
elected at random, in proportion to their stake.

While some protocols such as Ouroboros elect \emph{exactly one} leader, there
are protocols, in both the work and stake setting, such as Bitcoin and Ouroboros
Praos, which can elect multiple leaders for a particular time slot. However, the
final blockchain forms a \emph{sequence}, and hence only one block survives among multiple competing blocks.
As will shortly become clear, our analysis is not sensitive to such nuances.

Proof-of-stake is often criticized for its lack of egalitarianism. The rationale
is that, in proof-of-stake, the more money one stakes, the more money one is
able to generate. Thus, the \emph{rich get richer}, and that is precisely the
\emph{opposite} of egalitarianism. Additionally, in proof-of-stake systems, the
money owners could constitute a \emph{closed rich club} in which only they and
their friends have access to stake. In contrast, this argument claims,
proof-of-work is naturally egalitarian in the following manner. Everyone is paid
not according to the money the own and stake, but according to the computational
power they put to work. In this case, a closed rich club cannot be formed,
because computational power is a very \emph{natural} thing and cannot be
exclusively owned by the few like money can. In the worst case, one can still
have a minuscule chance of generating a block with computational power by mining
with pencil and paper~\cite{paper-mining}, or by building one's own computer. On
the other hand, if one were to be deprived from owning money by a closed system
of money owners, there is no natural process for them to generate it on their
own.

While this line of thought at first seems agreeable, we will show that this
is not the case in an open market. In fact, as we will soon see with concrete
data, stake-based systems are much more egalitarian than work. It is instructive
to dispell the above argument intuively before we support our position with
data. First of all, the argument that money can be exclusively owned, but
computational power cannot, is misguided. This may only be true in the case of
a peculiar oligopoly in which a small permissioned faction of parties mutually
agrees to only ever allow each other access to stake and never sell to
outsiders, despite external demand. However, in an open market, both money and
computational power can be freely purchased with money, and, in fact, any
non-negligible amount of computational power must necessarily be purchased on
the open market. Given that both money and computational power are purchasable,
the question comes down to how much money one needs to spend, investing either
in technology (and purchasing machines which generate computational power) or in
financial capital (and purchasing cryptocurrency which one can put to stake), in
order to maximize the money generated through a cryptocurrency's block
generation mechanisms. The amount of cryptocurrency generated by a given
investment can be concretely measured and compared, thus the question can now be
analyzed quantitatively and answered concretely.
