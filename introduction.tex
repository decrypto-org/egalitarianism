%!TEX root = egalitarianism.tex

\section{Introduction}

Bitcoin~\cite{bitcoin} is the first and most successful \emph{cryptocurrency} to
date. It introduced a cryptographic consensus protocol in
which \emph{transactions} are organized into \emph{blocks} which are put in a
globally agreed sequence, the \emph{blockchain}, despite the presence of adversaries. Since its
inception, a plethora of other cryptocurrencies have sprung into existence, each
claiming its own features.

There are two ways in which a cryptocurrency can choose to achieve consensus in
the decentralized setting:
\emph{proof-of-work}~\cite{C:DwoNao92} and
\emph{proof-of-stake}~\cite{C:KRDO17}. The consensus mechanism defines the
mandates of block generation. In the case of proof-of-work, where block
generation is called \emph{mining}, blocks are produced by \emph{miners}
expending computational power to solve cryptographic puzzles. On the other hand,
in the case of proof-of-stake, where block generation is called \emph{minting},
blocks are produced by \emph{minters} who ``stake'' their coins. In both cases,
during honest protocol executions, a leader is drawn at regular intervals at random from the
miners or minters population, with a probability of selection proportional to
their computational power or stake respectively.

Block generators are
incentivized to produce blocks by receiving a \emph{reward} for each block they
successfully produce and which is subsequently adopted in the resulting
blockchain.
% These rewards follow various schedules that are designed based on the
% macroeconomic desiderada envisioned by the architects of the cryptocurrency. For
% example, the rate of coin production is \emph{halved} every $210\,000$ blocks in
% Bitcoin. Ethereum and Litecoin follow similar schedules. On the contrary,
% Monero has a \emph{smooth emission} schedule in which the rewards are gradually
% reduced at every new block generated. The question of what this schedule should
% be can have significant impact on the variance of stake ownership after an
% execution of a sufficient number of protocol rounds~\cite{equitability}.
In this paper, we focus on the
comparison of the \emph{expected} returns of investors with different purchasing
power. The central economic question pertains to cryptocurrency \emph{egalitarianism}.
In an ideal world, investing a certain amount of capital to mining or minting
should allow a participant to receive rewards proportional to their capital;
that is, a \emph{poor dollar} should have equal participatory power as a \emph{rich dollar}.
However, this is far from true with most cryptocurrencies today.

\dionyziz{Fix comma typesetting in monetary values above (should not have a space after the comma)}

Though the term \emph{egalitarianism} has been left undefined, several
cryptocurrencies have claimed to achieve more egalitarianism than others.
However, lacking a comparable metric, the question of whether some
cryptocurrencies are more \emph{egalitarian} than others remains open.

In this paper, we put forth the first economic definition of egalitarianism
and provide a quantitative metric which can be practically measured and used
to compare different cryptocurrencies. Our definition is generic and can be
applied to any cryptocurrency.
Using our model, we then measure the egalitarianism of
four indicative proof-of-work-based
cryptocurrencies: Bitcoin, Litecoin, Ethereum, and Monero. Bitcoin was chosen as
a baseline for comparison as the first and most successful cryptocurrency to
date. Ethereum is the most promising ``altcoin,'' as cryptocurrencies beyond
Bitcoin came to be known, and is currently the highest decentralized cryptocurrency by market cap after Bitcoin. Litecoin and Monero, although not next by market cap,
make claims that they are more egalitarian because of their design.
We assess their claims and find them in agreement with our data.
We are the first to perform such concrete economic comparisons.
%  their mining puzzles are
% based on hash functions which are claimed to be memory-hard. Memory-hardness
% has the goal of making it costly to perform large-scale custom hardware attacks
% by requiring large amounts of memory, and hence are claimed to yield more
% egalitarian cryptocurrencies.
On the proof-of-stake side, as will soon become clear, egalitarian behavior is
similar across all coins regardless of implementation. Therefore, it suffices to
perform a case study of an indicative proof-of-stake currency. We study the case of a
pure proof-of-stake cryptocurrency, Cardano~\cite{C:KRDO17}, as well as a hybrid
proof-of-work/proof-of-stake cryptocurrency, Decred. We find that proof-of-stake
coins are more egalitarian than proof-of-work coins.

\dionyziz{Ensure Monero and Litecoin are more egalitarian than Bitcoin and Ethereum as claimed in the above paragraph}

\noindent
\textbf{Our contributions.}
Our contributions are summarized as follows:

\begin{enumerate}
  \item We define an exact economic measure of cryptocurrency
        \emph{egalitarianism}.
        To do this, we first define the \emph{egalitarian curve} of a
        cryptocurrency from which we extract the measure.
  \item We measure and compare the egalitarianism curve and egalitarianism of
        four indicative proof-of-work cryptocurrencies, one representative
        proof-of-stake cryptocurrency, and a hybrid cryptocurrency, using
        current market data.
  \item We show that proof-of-stake is, perhaps unexpectedly, perfectly
        egalitarian.
\end{enumerate}

Our paper is structured as follows. We begin by reviewing related work and preliminaries in
Sections~\ref{sec:related} and~\ref{sec:preliminaries}. We put forth our definition for the egalitarian curve
and egalitarianism of a cryptocurrency and motivate its intuition in
Section~\ref{sec:definition}. In Section~\ref{sec:experiments} we collect
empirical data for several cryptocurrencies of interest. We present our
conclusions in Section~\ref{sec:conclusion}.
