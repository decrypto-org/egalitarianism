%!TEX root = egalitarianism.tex

\section{Introduction}

In 2008, Satoshi Nakamoto proposed Bitcoin~\cite{bitcoin}, the first and most
successful \emph{cryptocurrency} to date. Bitcoin introduced a cryptographic
consensus protocol in which \emph{transactions} are organized into
\emph{blocks} which are put in a globally agreed sequence, the
\emph{blockchain}, despite the presence of adversaries and without the need of
any setup or identity system. Since its inception, a plethora of alternative
cryptocurrencies, or ``altcoins'', have sprung into existence, each claiming
its own features.

A major thread of research has focused on the mandates of block generation,
specifically the mechanism of identifying the party responsible for producing a
new, valid block at any point. Bitcoin, as well as the majority of altcoins,
employs \emph{proof-of-work}~\cite{C:DwoNao92}, where block generation is
called \emph{mining} and blocks are produced by \emph{miners} who expend
computational power to solve cryptographic puzzles. On the other hand, the most
prominent alternative mechanism is \emph{proof-of-stake}. In proof-of-stake
block generation is called \emph{minting} and blocks are produced by
\emph{minters} who ``stake'' their coins, \ie users who own a set of coins and
use them to participate in the consensus protocol. Intuitively, in both cases a
leader is drawn at regular intervals at random from the block generators
population, with a probability of selection proportional to their computational
power or stake respectively.

Block generators are incentivized to produce blocks by receiving a
\emph{reward} for each block they successfully produce and which is
subsequently adopted in the resulting blockchain.
% These rewards follow various schedules that are designed based on the
% macroeconomic desiderada envisioned by the architects of the cryptocurrency. For
% example, the rate of coin production is \emph{halved} every $210\,000$ blocks in
% Bitcoin. Ethereum and Litecoin follow similar schedules. On the contrary,
% Monero has a \emph{smooth emission} schedule in which the rewards are gradually
% reduced at every new block generated. The question of what this schedule should
% be can have significant impact on the variance of stake ownership after an
% execution of a sufficient number of protocol rounds~\cite{equitability}.
In this paper, we consider the block generators as investors and focus on the
comparison of the \emph{expected} returns of investors with different
purchasing power. The central economic property which arises is that of
cryptocurrency \emph{egalitarianism}. In an ideal world, investing a certain
amount of capital to produce blocks should result in rewards proportional to
that capital; that is, both a \emph{poor} investor and a \emph{rich} investor
should receive returns in proportion to their investment. In our point of view,
wealthy investors should not be rewarded
with disproportionate rewards and everybody should have equal opportunity to both participate
and earn rewards. However, this is far from true with most cryptocurrencies
today.

Until now, the term \emph{egalitarianism} has been left undefined, although
several cryptocurrencies claim to be more egalitarian than others.
Lacking a comparable metric, the question of whether some cryptocurrencies are
more \emph{egalitarian} than others has not been previously studied. In this paper, we
put forth the first definition of egalitarianism, in a way which is
generic and can be applied to any cryptocurrency.
Our definition provides a quantitative metric which can be practically measured and used
to compare different cryptocurrencies.
Using our model, we measure the egalitarianism of
four indicative proof-of-work-based
cryptocurrencies: Bitcoin, Litecoin~\cite{lee2011litecoin}, Ethereum~\cite{wood2014ethereum}, and Monero~\cite{van2013cryptonote}. Bitcoin, being the first and most successful cryptocurrency to date, was chosen as
the baseline of comparison. Ethereum is the most promising ``altcoin'' and is currently the largest decentralized cryptocurrency by market cap after Bitcoin\footnote{All references to market cap in this paper are made according to \url{https://coinmarketcap.com} [January 2019].}.
Litecoin and Monero, although not next by market cap,
make claims~\cite{van2013cryptonote,mcmillan2013} of increased egalitarianism because of their design.
We assess their claims and find them in agreement with our data, thus presenting for the first time economic comparisons which quantify them precisely.
%  their mining puzzles are
% based on hash functions which are claimed to be memory-hard. Memory-hardness
% has the goal of making it costly to perform large-scale custom hardware attacks
% by requiring large amounts of memory, and hence are claimed to yield more
% egalitarian cryptocurrencies.
On the proof-of-stake side, as will soon become clear, egalitarian
behavior is similar across all coins regardless of implementation. Therefore,
it suffices to perform a case study of indicative proof-of-stake protocols.
We study the case of proof-of-stake, applied on a protocol consistent with Ouroboros~\cite{C:KRDO17},
as well as a hybrid proof-of-work/proof-of-stake cryptocurrency,
Decred\footnote{\url{https://www.decred.org/}}. We find that proof-of-stake coins can be perfectly
egalitarian, contrary to their proof-of-work counterparts, provided they are parameterized correctly.

\noindent
\textbf{Our Contributions and Roadmap.}
This work provides a quantitative evaluation of cryptocurrency egalitarianism.
To the best of our knowledge this is the first work to provide a
treatment of this property and acts as the foundation for comparing
cryptocurrency fairness when it comes to reward distribution.
Specifically, the contributions of our research are summarized as follows:

\begin{enumerate}
  \item We define an exact measure of cryptocurrency
        \emph{egalitarianism}; to do this, we first define the \emph{egalitarian curve} of a
        cryptocurrency from which we extract the measure.
  \item We measure and compare the egalitarian curve and egalitarianism of
        four indicative proof-of-work cryptocurrencies, one representative
        proof-of-stake cryptocurrency, and a hybrid cryptocurrency, using
        current market data.
  \item We show that proof-of-stake, when correctly parameterized, is, perhaps unexpectedly, perfectly
        egalitarian.
\end{enumerate}

The rest of this paper is structured as follows. We begin by reviewing related
work and preliminaries in Sections~\ref{sec:related}
and~\ref{sec:preliminaries}. Next, we put forth our definition for the
egalitarian curve and egalitarianism of a cryptocurrency and motivate its
intuition in Section~\ref{sec:definition}. In Section~\ref{sec:experiments} we
present empirical data for several cryptocurrencies of interest and evaluate
the most popular cryptocurrencies under our model, in order to deduce whether
previous intuitive claims are indeed correct. Finally, the conclusions of our research
are drawn in Section~\ref{sec:conclusion}.
