%!TEX root = egalitarianism.tex

\section{Introduction}

Bitcoin~\cite{bitcoin} is the first and most successful \emph{cryptocurrency} to
date. It introduced a cryptographic consensus protocol in
which \emph{transactions} are organized into \emph{blocks} which are put in a
globally agreed sequence, the \emph{blockchain}, despite the presence of adversaries. Since its
inception, a plethora of other cryptocurrencies have sprung into existence, each
claiming its own features.

There are two ways in which a cryptocurrency can choose to achieve consensus in
the decentralized setting:
\emph{proof-of-work}~\cite{C:DwoNao92} and
\emph{proof-of-stake}~\cite{C:KRDO17}. The consensus mechanism defines the
mandates of block generation. In the case of proof-of-work, where block
generation is called \emph{mining}, blocks are produced by \emph{miners}
expending computational power to solve cryptographic puzzles. On the other hand,
in the case of proof-of-stake, where block generation is called \emph{minting},
blocks are produced by \emph{minters} who ``stake'' their coins. In both cases,
during honest protocol executions, a leader is drawn at regular intervals at random from the
miners or minters population, with a probability of selection proportional to
their computational power or stake respectively.

The leader election is a central part of each blockchain consensus algorithm. In
the proof-of-work case, miners attempt to solve the proof-of-work inequality, in
which they brute-force search for some string \textsc{nonce} such that
$H(\textsc{previd} || \textsc{data} || \textsc{nonce}) \leq T$. In this
inequality, $\textsc{previd}$ and $\textsc{data}$ are constants related to
blockchain maintenance and $T$ is a (relatively) small number called
the \emph{difficulty target}, which typically remains constant for periods of several
consecutive blocks called \emph{epochs} (for example, in Bitcoin, epochs are
2016 blocks long). Most importantly for our purposes, in this equation $H$ is a
cryptographically secure function. Because the search for solutions is
brute-forced, the expected number of solutions found by a given miner in a given
timeframe is proportional to the number of evaluations of the hash function $H$
she can obtain in this timeframe.

\dionyziz{Verify that modifying the difficulty retains the shape of our curves}

In the proof-of-stake case, a minter is selected in proportion to the stake they
hold, which is to say proportionally to the amount of money they own. There are
two flavours of this process. In the first flavour, all money owned automatically participates in the leader election
process. This is the case for the various variations of Ouroboros~\cite{ouroboros}
and Ethereum Casper~\cite{casper}. In the second flavour, the stake has to opt-in to participate in
the election by a special process, such as purchasing a so-called \emph{ticket}. This is the case for cryptocurrencies such as
Decred~\cite{decred}. Among those participating in the election, a leader is
elected at random, in proportion to their stake.

% While some protocols such as Ouroboros elect \emph{exactly one} leader, there
% are protocols, in both the work and stake setting, such as Bitcoin and Ouroboros
% Praos, which can elect multiple leaders for a particular time slot. However, the
% final blockchain forms a \emph{sequence}, and hence only one block survives among multiple competing blocks.
% As will shortly become clear, our analysis is not sensitive to such nuances.
%
Block generators are
incentivized to produce blocks by receiving a \emph{reward} for each block they
successfully produce and which is subsequently adopted in the resulting
blockchain.
% These rewards follow various schedules that are designed based on the
% macroeconomic desiderada envisioned by the architects of the cryptocurrency. For
% example, the rate of coin production is \emph{halved} every $210\,000$ blocks in
% Bitcoin. Ethereum and Litecoin follow similar schedules. On the contrary,
% Monero has a \emph{smooth emission} schedule in which the rewards are gradually
% reduced at every new block generated. The question of what this schedule should
% be can have significant impact on the variance of stake ownership after an
% execution of a sufficient number of protocol rounds~\cite{equitability}.
In this paper, we focus on the
comparison of the \emph{expected} returns of investors with different purchasing
power. The central economic question pertains to cryptocurrency \emph{egalitarianism}.
In an ideal world, investing a certain amount of capital to mining or minting
should allow a participant to receive rewards proportional to their capital.
However, this is not the case with most cryptocurrencies. Specifically, in
the case of mining, there exist various tiers of mining hardware which have
vastly different performance. If one were to mine using a CPU, a GPU, an FPGA,
or specialized hardware such as ASICs, it would make a tremendous difference in
their success rate and, in turn, return on investment, despite the pricing of
the respective hardware not having a similarly vast difference in cost of
purchase. Though the term \emph{egalitarianism} has been left undefined, several
cryptocurrencies have claimed to achieve more egalitarianism than others.
However, lacking a comparable metric, the question of whether some
cryptocurrencies are more \emph{egalitarian} than others remains open.

In this paper, we focus our research on four proof-of-work-based
cryptocurrencies: Bitcoin, Litecoin, Ethereum, and Monero. Bitcoin was chosen as
a baseline for comparison as the first and most successful cryptocurrency to
date. Ethereum is the most promising ``altcoin,'' as cryptocurrencies beyond
Bitcoin came to be known. Litecoin and Monero, although not next by market cap,
do make claims that they are more egalitarian because their mining puzzles are
based on hash functions which are claimed to be memory-hard. Memory-hardness
has the goal of making it costly to perform large-scale custom hardware attacks
by requiring large amounts of memory, and hence are claimed to yield more
egalitarian cryptocurrencies.

On the proof-of-stake side, as will soon become clear, egalitarian behavior is
similar across all coins regardless of implementation. Therefore, it suffices to
perform a case study of an indicative proof-of-stake currency. We study the case of a
pure proof-of-stake cryptocurrency, Cardano~\cite{C:KRDO17}, as well as a hybrid
work/stake cryptocurrency, Decred.

\noindent
\textbf{Our contributions.}
Our contributions are summarized as follows:

\begin{enumerate}
  \item We define an exact economic measure of cryptocurrency
        \emph{egalitarianism}.
        To do this, we first define the \emph{egalitarian curve} of a
        cryptocurrency from which we extract the measure.
  \item We measure and compare the egalitarianism curve and egalitarianism of
        four indicative proof-of-work cryptocurrencies, one representative
        proof-of-stake cryptocurrency, and a hybrid cryptocurrency, using
        current market data.
  \item We show that proof-of-stake is, perhaps unexpectedly, perfectly
        egalitarian.
\end{enumerate}

Our paper is structured as follows. We begin by reviewing related work in
Section~\ref{sec:related}. We put forth our definition for the egalitarian curve
and egalitarianism of a cryptocurrency and motivate its intuition in
Section~\ref{sec:definition}. In Section~\ref{sec:experiments} we collect
empirical data for several cryptocurrencies of interest. We present our
conclusions in Section~\ref{sec:conclusion}.

\section{Related work}\label{sec:related}
The macro and microeconomics of blockchain design have
been studied from several perspectives but remain an active area of research
which is not yet fully understood. The topic of incentivizing participants to
generate blocks according to the honest protocol has been explored for both
proof-of-work and proof-of-stake. Proof-of-work protocols such as Bitcoin were
formalized in the Bitcoin Backbone~\cite{backbone1,backbone2} papers and
follow-up works~\cite{pass}. The seminal work of Selish Mining~\cite{selfish}
showed that the honest behavior is not incentive-compatible in Bitcoin, but the
protocol can be modified to behave that way. On the other hand, restricted
renditions of Bitcoin have been shown to be incentive-compatible in limited
models~\cite{tselekounis-kiayias}. Proof-of-stake systems such as Ouroboros
have been designed from the ground up to be
incentive-compatible~\cite{ouroboros}. The question of how to incentivize
parties to conduct pool formation into the desired number of pools, or groups of minters, was studied
in~\cite{stouka-koutsoupias-kiayias}.

The above works study the incentives of blockchain systems from the designer's
point of view so that participants do not deviate from the prescribed protocol.
A related question is how \emph{fair} the protocol is to participants
themselves, and in particular to honest participants. The Backbone and Selfish
Mining works include attacks in which an adversary can strategically harm
\emph{chain quality}, ensuring the number of blocks and, in turn, the respective
rewards, are disproportionate to their contributed computational power, thereby
harming fairness against honest participants. Fruitchains~\cite{fruitchains}
proposes a protocol which solves this problem.

\dionyziz{fill-in above citations}
\dionyziz{memory-hard functions related work}
\dionyziz{equitability related work}
