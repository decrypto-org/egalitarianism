\section{Related work}\label{sec:related}
The macro and microeconomics of blockchain design have been studied from
several perspectives but remain an active area of research with a number of open questions.
Incentives for block generation
according to the honest protocol have been explored for both proof-of-work and
proof-of-stake.

Proof-of-work protocols such as Bitcoin were formalized in the Bitcoin
Backbone~\cite{EC:GarKiaLeo15,C:GarKiaLeo17} papers and follow-up
works~\cite{pass2017analysis}. The seminal work of Selish
Mining~\cite{FC:EyaSir14,FC:SapSomZoh16} showed that the honest behavior is not
incentive-compatible in Bitcoin, but the protocol can be modified to behave
that way, such as the restricted rendition of Bitcoin
in~\cite{kiayias2016blockchain}.  Proof-of-stake protocols such as
Ouroboros~\cite{C:KRDO17} can be designed from the ground up to be
incentive-compatible, \eg the question of how to incentivize parties to conduct
pool formation into the desired number of pools, or groups of minters, was
studied in~\cite{bkks2018}.

The above works study the incentives of blockchain systems from the designer's
point of view so that participants do not deviate from the prescribed protocol.
A related question is how \emph{fair} the protocol is to participants
themselves, and in particular to honest participants. The Backbone and Selfish
Mining works include attacks in which an adversary can strategically harm
\emph{chain quality}, ensuring the number of blocks and, in turn, the respective
rewards, are disproportionate to their contributed computational power, thereby
harming fairness against honest participants. Fruitchains~\cite{PODC:PasShi17}
proposes a protocol which solves this problem, but, although
these works study incentives of block generation, it has been
assumed that handing out rewards in exact proportion to computational power is
``fair''.

\emph{Egalitarianism}, in the way considered in the paper, has been studied in
proof-of-work systems from a technological point of view with respect to
\emph{memory-hard functions} in~\cite{pebbling}\todo{memory-hard functions
related work citations (pebbling games - ask orestis)}.  However, the question
of whether computational power grows proportionally to capital invested, \ie
whether accumulated wealth results in more than proportional rewards, has not
been previously studied. Therefore, our work aims at filling this gap by
studying the effects of economies of scale when applied to cryptocurrency coin
generation.

\dionyziz{verify monero ASIC-less claim above}

\noindent\textbf{Equitability of cryptocurrencies.}
An analysis of economic fairness of
blockchain reward functions was recently studied by Fanti \textit{et al.} in~\cite{fanti2018compounding}, which
introduces the notion of \emph{equitability}. This work studies the evolution
of a blockchain system after a series of rounds, putting forth the
---desirable--- property that stake ownership remains proportionally the same
both before and after the rewards of a fixed period of time have been awarded.
For example, if an
investor initially owns $20\%$ of the total funds, an equitable system ensures that they
own $20\%$ of the total funds eventually, \ie including
rewards. By studying the behaviour of the returns' \emph{variance} under the
randomness of executions, they show that the distribution of capital
follows a Pòlya process and propose a variance reduction mechanism
by the introduction of a \emph{geometric reward} function.
We note that, in contrast, our work studies the \emph{expectation} of rewards given different
initial capital.

This work makes two simplifying assumptions.
First, they assume that the \emph{expectation} of returns is proportional to the
invested capital. As we show in this work, this is only
approximately true for some proof-of-stake systems, specifically the perfectly egalitarian. For proof-of-work, they
correctly assume that the expectation of returns is proportional to computational
power expended, but do not study the relationship between computational power and
capital, which, as we show in this work, is far from linear. Second, they assume that
proof-of-work miners do not reinvest their proceeds in the
mining operation, while proof-of-stake minters do. This is the critical assumption which allows the authors to
conclude that proof-of-stake minting behaves like Pòlya's urn and
proof-of-work mining does not. As we show in this work, this
is highly unlikely, as this strategy is not profitable to proof-of-work
miners. In fact, empirical data show that large-scale miners pay for electricity
directly using their proceeds~\footnote{See:
\url{https://www.bloomberg.com/news/articles/2018-04-18/bitcoin-miners-facing-a-shakeout-as-profitability-becomes-harder}
[Online: December 2018]}. Hence, their model applies to both proof-of-work and
proof-of-stake systems, meaning that they are both equally unequitable.

We note that \emph{egalitarianism} and \emph{equitability} are orthogonal
notions. In fact, a proof-of-stake cryptocurrency can be perfectly egalitarian
and poorly equitable, \eg if it uses a constant or decreasing reward function.
Similarly, a cryptocurrency may not be egalitarian but maximally equitable, \eg
using a geometric reward function in conjuction with a proof-of-work function
which lends itself to economies of scale and allows the construction of
specialized hardware. Naturally, the ideal cryptocurrency would be both
\emph{perfectly egalitarian} and \emph{maximally equitable}, \eg by adopting
proof-of-stake under a geometric reward function.

\dionyziz{Cite Adam Hayes' work on the inherent worth of proof-of-work?}
