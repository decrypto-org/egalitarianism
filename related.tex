\section{Related work}\label{sec:related}
The macro and microeconomics of blockchain design have been studied from
several perspectives but remain an active area of research with a number of open questions.
Incentives for block generation
according to the honest protocol have been explored for both proof-of-work and
proof-of-stake.

Proof-of-work protocols such as Bitcoin were formalized in the Bitcoin
Backbone~\cite{EC:GarKiaLeo15,C:GarKiaLeo17} papers and follow-up
works~\cite{pass2017analysis}. The seminal work of Selfish
Mining~\cite{FC:EyaSir14}, see also \cite{FC:SapSomZoh16,kiayias2016blockchain} showed that
honest behavior is not incentive-compatible. Alternative reward sharing mechanisms
in the proof-of-stake  setting
make it feasible to behave better in terms of incentive compatibility for instance
Ouroboros~\cite{C:KRDO17} can be designed from the ground up to be a Nash
equilibrium under certain plausible conditions
and similarly, in the proof-of-work setting~\cite{PODC:PasShi17}.
The question of how to incentivize parties to conduct
pool formation into the desired number of pools, or groups of minters, was
studied in~\cite{bkks2018}.
%
%The above works study the incentives of blockchain systems from the designer's
%point of view so that participants do not deviate from the prescribed protocol.
%A related question is how \emph{fair} the protocol is to participants
%themselves, and in particular to honest participants. The Backbone and Selfish
%Mining works include attacks in which an adversary can strategically harm
%\emph{chain quality}, causing the number of blocks and, in turn, the respective
%rewards, to be disproportionate to their contributed computational power, thereby
%harming fairness against honest participants. Fruitchains
%proposes a protocol which solves this problem. In these works,
%handing out rewards in exact proportion to computational power is considered
%``fair.''

\emph{Egalitarianism} has been studied before in
proof-of-work systems from the perspective of
\emph{memory-hard functions} in~\cite{alwen2017depth,biryukov2016egalitarian},
under the premise that memory hardness provides egalitarianism in the sense
that the it can be used to argue that the
cost of one computational step will be roughly  the same irrespective of the
underlying
computational platform (typically ASIC vs. generic, cf.  \cite{biryukov2016egalitarian}).
The approach we take here instead, asking whether
computational power grows proportionally to capital invested, \ie
whether larger wealth results in more than proportional rewards,
  is more general and it has not
been previously studied to the best of our knowledge.
%Therefore, our work aims at filling this gap by
%studying the effects of economies of scale when applied to cryptocurrency
%generation.

\noindent\textbf{Equitability of cryptocurrencies.}
Fanti \textit{et al.} analyze economic blockchain fairness
in~\cite{equitability}, where they define \emph{equitability}. They study the
evolution of a system after a series of rounds, putting forth the property that
stake ownership remains in proportion \emph{before} and \emph{after} rewards
have been awarded.
By studying the behaviour of the returns' variance under the randomness of
executions, they introduce a geometric reward function and show its optimality
in terms of equitability.  Whereas their equitability metric jointly captures
the normalized variance of rewards for every user conditioned on their initial
resources (e.g., fraction of computational power for PoW), our egalitarianism
metric instead captures the population-wide variation of best-case expected
returns for an initial capital distribution among participants. In other words,
our randomness is over the initial distribution of wealth, whereas theirs is
over the evolution of a single blockchain execution.
In our work, we show that
computational power is not proportional to the invested capital, and hence the
analogy between proof-of-work computational power and proof-of-stake capital
breaks down, and a more detailed study is needed. Additionally, we remark that
proof-of-work miners also reinvest their proceeds in the mining operation,
albeit slowly, as proof-of-stake minters do. For example, empirical data show
that large-scale miners pay for electricity using their
proceeds~\cite{kharif2018}. Hence, both mining and minting follow Pólya
processes as modelled by their paper.
Regardless, \emph{egalitarianism} and \emph{equitability} are orthogonal. A
cryptocurrency can be perfectly egalitarian and poorly equitable and vice versa.
It is possible to obtain a cryptocurrency both egalitarian and equitable by
adopting correctly parameterized proof-of-stake under a geometric reward
function.
