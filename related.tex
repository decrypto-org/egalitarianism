\section{Related work}\label{sec:related}
The macro and microeconomics of blockchain design have
been studied from several perspectives but remain an active area of research
which is not yet fully understood. The topic of incentivizing participants to
generate blocks according to the honest protocol has been explored for both
proof-of-work and proof-of-stake. Proof-of-work protocols such as Bitcoin were
formalized in the Bitcoin Backbone~\cite{EC:GarKiaLeo15,C:GarKiaLeo17} papers and
follow-up works~\cite{pass}. The seminal work of Selish Mining~\cite{selfish}
showed that the honest behavior is not incentive-compatible in Bitcoin, but the
protocol can be modified to behave that way. On the other hand, restricted
renditions of Bitcoin have been shown to be incentive-compatible in limited
models~\cite{tselekounis-kiayias}. Proof-of-stake systems such as Ouroboros
have been designed from the ground up to be
incentive-compatible~\cite{ouroboros}. The question of how to incentivize
parties to conduct pool formation into the desired number of pools, or groups of minters, was studied
in~\cite{stouka-koutsoupias-kiayias}.

The above works study the incentives of blockchain systems from the designer's
point of view so that participants do not deviate from the prescribed protocol.
A related question is how \emph{fair} the protocol is to participants
themselves, and in particular to honest participants. The Backbone and Selfish
Mining works include attacks in which an adversary can strategically harm
\emph{chain quality}, ensuring the number of blocks and, in turn, the respective
rewards, are disproportionate to their contributed computational power, thereby
harming fairness against honest participants. Fruitchains~\cite{fruitchains}
proposes a protocol which solves this problem.
While these works study incentives of block generation, it has been
assumed that handing out rewards in exact proportion to computational power is
``fair''. However, the question of whether computational power grows proportionally
to capital invested has not been previously studied. As such, the notion of
fairness in treating each dollar equally, regardless of whether it comes from a
rich or poor fund, has not been previously studied.

Intuitively, we are studying the effects of economies of scale as they are
applied to block generation.
In proof-of-work systems, the question of \emph{egalitarianism} has been studied
from a technological point of view through the work on \emph{memory-hard functions}~\cite{pebbling}.

\dionyziz{verify monero ASIC-less claim above}
\dionyziz{fill-in above citations}
\dionyziz{memory-hard functions related work citations (pebbling games - ask orestis)}

An analysis of economic fairness pertaining to blockchain reward functions has recently
been studied in the work of~\cite{equitability}, where they propose the notion
of \emph{equitability}. In their work, they study the evolution of a blockchain system
after a series of rounds and they put forth the desirable property that stake ownership
after the rewards of a fixed period of time have been handed out is proportional
to stake ownership before the rewards have been handed out. For example, if an
investor initially owns $20\%$ of the total funds before the game begins, they
should own $20\%$ of the total funds when the game ends, which now includes
rewards. They study how the \emph{variance} of these returns behaves under the
randomness of executions. They show that the distribution of capital after
execution follows a Pòlya process and propose a mechanism to reduce its
variance by the introduction of a \emph{geometric reward} function.
In contrast, our work studies the \emph{expectation} of rewards given different
initial capital.
In their work, they make two simplifying assumptions.
First, they assume that the \emph{expectation} of returns is proportional to
capital invested for proof-of-stake. As we show in this work, this is only
approximately true for some proof-of-stake systems (those that are perfectly egalitarian). For proof-of-work, they
correctly assume that the expectation of returns is proportional to computational
power expended, but do not study the relationship between computational power and
capital, which, as we show in this work, is far from linear. Secondly, they assume that
proof-of-work miners do not reinvest their proceeds back in the
mining operation, while proof-of-stake minters do. This is the critical assumption which allows the authors to incorrectly
conclude that proof-of-stake minting behaves like Pòlya's urn, while
proof-of-work mining does not. As we show in this work, this
is highly unlikely, as such a strategy would not be profitable to proof-of-work
miners. In fact, empirical data shows that large-scale miners pay for electricity
directly using their proceeds~\cite{is-this-even-true}. Hence, their mathematical analysis
using a Pòlya process is correct, but, contrary to their claims, their model applies to proof-of-work
and proof-of-stake systems alike, meaning that they are both equally unequitable. Regardless,
\emph{egalitarianism}, which we treat in this work, and \emph{equitability}, which
they treat in their work, are two orthogonal notions. A
cryptocurrency can be perfectly egalitarian and poorly equitable if it uses a
constant or decreasing reward function and is proof-of-stake. On the other hand,
a cryptocurrency can be poorly egalitarian but maximally equitable if it uses a
geometric reward function in conjuction with a proof-of-work function which
lends itself to economies of scale and allows the construction of specialized
hardware. The reader is invited to contrast the two notions, which measure
block generation fairness from a different point of view. Naturally, the ideal cryptocurrency is both \emph{perfectly egalitarian}
and \emph{maximally equitable}. Designers of such a currency would adopt
proof-of-stake under a geometric reward function.

\dionyziz{Ensure we illustrate the above claim -- non-reinvestment is non-profitable}
\dionyziz{Investigate whether large-scale miners pay for their electricity in bitcoin and find a relevant source}
