\section{Conclusion}\label{sec:conclusion}

In this work we explore the notion of \emph{egalitarianism} of
cryptocurrencies. Although this notion has long been discussed, this is the
first attempt at a formal definition, which allows us to concretely argue about
the egalitarianism of various existing systems.

Indeed, the results of our experimental simulations are very optimistic in
terms of usability of our metric, as they provide us with concrete figures
which establish the egalitarianism of some of the most popular
cryptocurrencies.

The most exciting result arises from the comparison between the proof-of-work
and proof-of-stake mechanisms. Although, up until now, intuition led a large
part of the blockchain community to argue in favour of proof-of-work systems,
claiming better egalitarianism in this setting, our results argue the exact
opposite: in fact, it is proof-of-stake systems which are more egalitarian.

Our work provides the first step towards establishing a concrete framework of
evaluation of egalitarianism in the cryptocurrency ecosystem. Future work will
be focused in both evaluating more existing cryptocurrencies, as well as
investigating variations of consensus mechanisms, such as delegated
proof-of-stake.
