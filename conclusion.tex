\section{Conclusion}\label{sec:conclusion}

In this work we explore the notion of \emph{egalitarianism} of
cryptocurrencies. Although this notion has long been discussed, we are the first
to give a definition, which allows us to concretely argue about the
egalitarianism of various existing systems.

The results of our experimental simulations are very optimistic in
terms of usability of our metric, as they provide concrete figures
which measure the egalitarianism of several popular cryptocurrencies.
The most unexpected result arises from the comparison between the proof-of-work
and proof-of-stake mechanisms. Although blockchain folklore argued in favour of
proof-of-work systems in terms of egalitarianism,
our results show that, in fact, it is proof-of-stake systems which are more
egalitarian in our model.

Our work provides the first step towards establishing a concrete framework of
egalitarianism evaluation in the cryptocurrency ecosystem. Future work will
focus in evaluating more existing cryptocurrencies and
investigating variations of consensus mechanisms such as delegated
proof-of-stake. Additionally, we leave for future work the treatment of more
complex economical models
of the mining game such as dynamic systems and adversarial strategies, as well
as economies of scale in the multitude of parameters we have ignored, such as
electricity bulk pricing. We conjecture the consideration of such parameters
will exacerbate the gap between proof-of-work and proof-of-stake which we have
illustrated in this work.

Finally, we remark that neither proof-of-work nor proof-of-stake blockchains
are politically egalitarian systems, in which the ideal of
\emph{one human one vote} is attained. Instead, at best,
\emph{one coin one vote} is attained in the case of well-parameterized
proof-of-stake systems. Thus, as illustrated in this paper, blockchain systems are, for
the time being, plutocratic. Whether decentralized decision making in which
each human is allocated one vote is possible remains an open question.
In such a system, the egalitarian curve would be strictly
decreasing; however, our results, especially Lemma~\ref{lem:sybil}, hint towards our
conjecture that such systems are impossible in a Sybil-resilient setting.
